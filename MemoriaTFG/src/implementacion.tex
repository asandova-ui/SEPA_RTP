\section{Diseño e Implementación}
\label{sec:DisenoImplementacion}
Aquí se detalla el proceso de diseño e implementación del servidor HTTP...

\subsection{Arquitectura del servidor HTTP}
\label{subsec:ArquitecturaHTTP}
\subsubsection{Estructura y funcionamiento}
\label{subsubsec:EstructuraFuncionamiento}
El servidor se diseñó para manejar solicitudes y respuestas HTTP de manera eficiente...

\subsubsection{Protocolos y estándares implementados}
\label{subsubsec:Protocolos}
Se implementaron protocolos como HTTP/1.1 y se consideraron estándares de seguridad...

\subsection{Emulación del prototipo \textit{Request To Pay}}
\label{subsec:EmulacionRTP}
\subsubsection{Diseño de las funcionalidades}
\label{subsubsec:DisenoFuncionalidades}
Se replicaron las funcionalidades clave del sistema \textit{Request To Pay}...

\subsubsection{Modelos de procesamiento de solicitudes}
\label{subsubsec:ProcesamientoSolicitudes}
La lógica de negocio se diseñó para procesar solicitudes de pago...

\subsection{Herramientas de desarrollo}
\label{subsec:HerramientasDesarrollo}
\subsubsection{Lenguajes y frameworks}
\label{subsubsec:LenguajesFrameworks}
Se utilizó [especificar lenguaje/framework, e.g., Node.js] para el desarrollo...

\subsubsection{Pruebas y validación}
\label{subsubsec:PruebasValidacion}
Se emplearon herramientas como Postman para validar la funcionalidad del servidor...
