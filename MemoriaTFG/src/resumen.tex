\phantomsection
\addcontentsline{toc}{section}{Resumen}

\thispagestyle{empty}
\selectlanguage{spanish}

\section*{Resumen}
Este Trabajo de Fin de Grado explora el potencial del esquema \textbf{SEPA\cite{ecb_sepa} \textit{Request‑to‑Pay} (SRTP)} como alternativa de próxima generación al \emph{SEPA Direct Debit} (SDD). Parte de la premisa de que la domiciliación bancaria actual limita la inmediatez, la trazabilidad y el control del pagador. Para validar la propuesta se ha diseñado y construido un prototipo end‑to‑end que reproduce el ciclo completo de una petición de pago: creación, presentación, decisión, ejecución y cierre. El prototipo incluye un back‑end orientado a eventos, una interfaz web en tiempo real y un banco de pruebas que emula a los participantes del ecosistema SRTP. Sobre esta base se han realizado pruebas funcionales y de rendimiento, escalabilidad y robustez frente a fallos. Los resultados confirman que SRTP puede reducir los tiempos de conciliación y mejorar la visibilidad de las operaciones sin comprometer la seguridad. Finalmente, se discuten las implicaciones de negocio, los requisitos regulatorios y las líneas de trabajo futuro para su industrialización.

\newpage

\selectlanguage{english}
\section*{Abstract}
This Final Degree Project investigates the potential of the \textbf{SEPA\cite{ecb_sepa} \textit{Request‑to‑Pay} (SRTP)} scheme as a next‑generation alternative to traditional \emph{SEPA Direct Debit} (SDD). The study starts from the premise that current direct‑debit processes limit immediacy, traceability and payer control. To validate the proposal, a full end‑to‑end prototype has been designed and built that reproduces the entire payment‑request life‑cycle: creation, presentation, decision, execution and settlement. The prototype comprises an event‑driven back‑end, a real‑time web interface and a test harness that emulates the SRTP ecosystem participants. Functional and performance tests were conducted, measuring scalability and fault tolerance. The results confirm that SRTP can shorten reconciliation times and improve operational visibility without sacrificing security. Business implications, regulatory requirements and future lines of work needed for industrialisation are also discussed.

