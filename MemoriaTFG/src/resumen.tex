\phantomsection
\addcontentsline{toc}{section}{Resumen}

\thispagestyle{empty}
\section*{Resumen}

Este Trabajo de Fin de Grado (TFG) explora cómo mejorar los sistemas de cobro en Europa, comparando el método tradicional de domiciliación bancaria (SEPA Direct Debit o SDD) con una solución más moderna: el esquema Request To Pay (RTP). El SDD, aunque muy usado, tiene limitaciones como procesos lentos, necesidad de documentos físicos y riesgos de devoluciones, lo que lo hace poco práctico para el mundo digital actual, donde se busca rapidez y simplicidad. En cambio, el RTP permite a quien cobra enviar una solicitud de pago digital que el pagador puede aceptar o rechazar al instante, haciendo el proceso más rápido, seguro y eficiente.

El objetivo del TFG ha sido crear un prototipo que simule cómo funcionaría un proveedor de RTP. Este prototipo, desarrollado con herramientas como Node.js y una base de datos sencilla, muestra cómo se pueden gestionar solicitudes de pago en tiempo real, desde su creación hasta su aprobación o rechazo. Las pruebas realizadas confirman que el sistema funciona bien y resuelve problemas del SDD, como la lentitud y la falta de control inmediato.

Este trabajo no solo demuestra que el RTP puede ser una alternativa útil para modernizar los pagos, sino que también abre la puerta a mejoras futuras, como hacerlo más seguro o conectarlo con bancos reales. En resumen, el prototipo es un paso hacia un sistema de cobros más ágil y adaptado a las necesidades de hoy, con potencial para cambiar cómo manejamos las transacciones en Europa.

