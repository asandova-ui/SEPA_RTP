\section{Introducción}
\label{sec:Introduccion}

Para comprender el entorno actual de los pagos en Europa, conviene arrancar por la \emph{Single Euro Payments Area} (SEPA): un espacio comunitario en el que todos los pagos en euros se rigen por los mismos estándares técnicos y normas operativas, de modo que enviar dinero de un país a otro resulta tan ágil y claro como una transferencia nacional. SEPA estableció protocolos de mensajería comunes, armonizó los plazos de liquidación y fijó reglas uniformes de protección al usuario, creando la base sobre la que se despliegan hoy los servicios de pago más innovadores.

En los últimos diez años, la digitalización de los servicios financieros ha cambiado por completo cómo particulares y empresas gestionan sus transacciones dentro de ese marco SEPA. Las transferencias instantáneas, las API abiertas de los bancos y el auge del comercio electrónico han disparado la demanda de procesos de cobro que sean sencillos, transparentes y en tiempo real. No obstante, los instrumentos de pago tradicionales —tarjetas, transferencias convencionales o domiciliaciones— nacieron en un contexto muy distinto y todavía arrastran limitaciones que penalizan tanto la experiencia de usuario como la eficiencia operativa.

Aquí es donde entra en juego \emph{Request-to-Pay} (RTP). Este servicio de mensajería permite al beneficiario enviar al pagador una solicitud de pago digital estructurada, con todos los detalles (importe, concepto, vencimiento), y recibir en segundos una respuesta —aceptación, rechazo o aplazamiento— antes de iniciar el movimiento de fondos. RTP no sustituye los métodos de pago existentes, sino que actúa como una capa de orquestación sobre la infraestructura SEPA (y, en especial, los pagos inmediatos) y los canales de banca online, facilitando la conciliación, reduciendo la fricción en el cobro y modernizando la experiencia tanto para empresas como para consumidores.

\subsection{Motivación}
\label{subsec:Motivacion}

La domiciliación bancaria regulada por el esquema \textit{SEPA Direct Debit} (\textbf{SDD})\footnote{\textit{SEPA Direct Debit} es el instrumento paneuropeo de cargo en cuenta regulado por el \emph{European Payments Council}.} desde 2014, sigue siendo el método principal para cobros recurrentes en España. No obstante, su estructura, pensada para un entorno de procesos \emph{offline}—genera hoy inconvenientes que chocan con las demandas de inmediatez, seguridad y experiencia de usuario fluida que caracterizan la economía digital actual.

\subsubsection{Ineficiencias operativas detectadas}

Tras analizar la operativa SDD nacional se han detectado una serie de ineficiencias resumidas en los siguientes \textbf{5} puntos:
\begin{enumerate}[label=\textbf{\arabic*.}, leftmargin=0.75cm]
  \item \textbf{Modelo off-line y necesidad de mandato físico}\\
        \begin{itemize}[leftmargin=0.45cm]
            \item El pago se inicia sin interacción \emph{online}; El deudor debe firmar y remitir \textbf{mandato SEPA}\footnote{Autorización que un deudor otroga a un acreedor para permitir el cobro automático de pagos mediante SDD.} y el acreedor debe custodiarlo durante el periodo de vida del contrato y hasta 14 meses tras la última transacción.
            \item No existe un estándar eMandate\footnote{Un eMandate es la versión digital de un mandato.} interoperable, cada banco utiliza su propio estándar.
        \end{itemize}
        \vspace{-0.1cm}
        \textit{Consecuencias}: Fricción en la venta digital, costes de back-office (archivado, auditorías y gestión) y riesgo legal si el mandato no aparece ante una devolución.

  \item \textbf{Derecho a devolución prolongado}\\
        \begin{itemize}[leftmargin=0.45cm]
            \item \emph{Cobro autorizado}: el deudor dispone de \textbf{8~semanas} para devolver sin causa (\emph{“no-questions-asked”}).%
            \item \emph{Cobro no autorizado}: hasta \textbf{13~meses} para reclamar si el banco emisor no puede demostrar mandato válido.%
        \end{itemize}
        \vspace{-0.1cm}
        \textit{Consecuencias}: gran incertidumbre sobre la firmeza del ingreso, reservas de liquidez, provisiones contables y alto nivel de \emph{friendly fraud}\footnote{servicio consumido y posterior devolución.}.

  \item \textbf{Ciclos de cobro lentos}\\
        \begin{itemize}[leftmargin=0.45cm]
            \item En esquema \textsc{Core}\footnote{CORE es el esquema estándar de domiciliación bancaria SEPA para pagos entre empresas y consumidores.}: envío al banco \textbf{D-5} para primera domiciliación y \textbf{D-2} para recurrencias; liquidación interbancaria \textbf{+2~días}.
            \item 6–8~días naturales entre petición y abono firme, incompatibles con venta inmediata o entrega digital.
        \end{itemize}
        \vspace{-0.1cm}
        \textit{Consecuencias}: Tesorería imprevisible (cash-flow) y riesgo de prestar servicio sin cobro confirmado.

  \item \textbf{Costes y complejidad de las R-transactions\footnote{ transacciones de rechazo, devolución o reembolso asociadas a pagos fallidos o no autorizados.}}\\
        \begin{itemize}[leftmargin=0.45cm]
            \item Existen distintos códigos de R-transactions y cada uno de ellos implica un flujo distinto, lo que dificulta la gestión.
        \end{itemize}
        \vspace{-0.1cm}
        \textit{Consecuencias}: Necesidad de equipos específicos de conciliación y recobro, lo que conlleva un coste directo y pérdida de productividad.

  \item \textbf{Ausencia de autorización fuerte (\textsc{SCA})\footnote{La SCA (Strong Customer Authentication) es un mecanismo de seguridad que exige verificar al usuario con al menos dos factores.}}\\
        \begin{itemize}[leftmargin=0.45cm]
            \item El SDD se basa en consentimiento previo, no aplica SCA al momento del cargo.
        \end{itemize}
        \vspace{-0.1cm}
        \textit{Consecuencias}: Mayor riesgo de cargos disputados y se desaprovechan métodos de identidad digital ya existentes.
\end{enumerate}

\paragraph{En conclusión.} Estas ineficiencias se traducen en (i) estructura de costes elevada por devoluciones y personal especializado; (ii) liquidez incierta—los ingresos se confirman con días de retraso y pueden desaparecer meses después—y (iii) freno a la economía digital online, incapaz de ofrecer experiencias de pago instantáneas comparables a tarjeta, monederos o Bizum.

\subsubsection{Oportunidad de un esquema Request-to-Pay}
El estándar \textit{SEPA Request-to-Pay} (\textbf{SRTP})\footnote{Iniciativa del European Payments Council que define un flujo de solicitud (\textit{request}) y aceptación de pagos en tiempo real, apoyado en mensajería \textbf{ISO 20022} (p.\,ej.\, \texttt{pain.013}/\texttt{pain.014}) y agnóstico respecto al instrumento de liquidación posterior.} se perfila como la evolución natural de la domiciliación SEPA. Sus ventajas técnicas frente al SDD son:

\begin{enumerate}[label=\alph*)]
  \item \textbf{Autenticación reforzada y consentimiento digital inmediato}\\
        El acreedor emite un request que el deudor aprueba en su banca o \emph{wallet} mediante \textsc{SCA}. Este gesto sustituye al mandato físico y genera una prueba electrónica de consentimiento, firmada y almacenada dentro del PSP del pagador.

  \item \textbf{Irrevocabilidad y mitigación de fraude \emph{post-servicio}}\\
        Tras la aceptación, el pago se realiza mediante SCT Inst\footnote{\textsc{SCT Inst}: transferencia inmediata SEPA con liquidación ≤10 s.}. Al no existir derecho de devolución automática, se elimina el \emph{friendly fraud} asociado a la devolución de recibos y se reducen provisiones por impago.

  \item \textbf{Liquidez \emph{real-time} y conciliación automática}\\
        La disponibilidad de fondos en \(\leq10\,\text{s}\) permite planificar tesorería al instante. Los identificadores de extremo a extremo y las referencias estructuradas ISO 20022 se transmiten sin perderse entre sistemas, de modo que la conciliación queda totalmente automatizada.

  \item \textbf{Simplificación operativa}\\
        Desaparecen las \textsc{R-transactions}, la custodia de mandatos y las tareas de back-office. El flujo se limita a dos mensajes (request y aceptación) y, opcionalmente, una transferencia instantánea, con clara trazabilidad extremo a extremo.

  \item \textbf{Flexibilidad comercial y costes reducidos}\\
        SRTP admite cobros únicos, recurrentes y fraccionados desde web o app vía enlaces profundos, QR o API, y al ser un pago mediante SCT Inst, las comisiones bancarias son muy inferiores a las de tarjeta o a las de gestión de devoluciones SDD.
\end{enumerate}

En síntesis, SRTP traslada las ventajas históricas de la domiciliación—bajo coste y cobertura paneuropea—al entorno digital e inmediato, resolviendo los puntos técnicos que hoy limitan la competitividad del SDD en España y en la zona SEPA.
\vspace{0.5cm}

\renewcommand{\arraystretch}{1.3}
\begin{tabular}{@{}p{4.5cm}p{4.1cm}p{4.1cm}@{}}
\toprule
\textbf{Aspecto} & \textbf{SDD} & \textbf{SRTP (+ SCT Inst)} \\ \midrule
Autorización & Mandato off-line & Consentimiento digital (\textsc{SCA}) en tiempo real \\ 
Plazo de devolución & 8\,semanas / 13\,meses & No aplica (operación irrevocable) \\ 
Disponibilidad de fondos & 5–8 días & Segundos (\textless10\,s) \\ 
Coste operativo & Alto (mandatos, \textsc{R-codes}) & Bajo (mensajería ISO 20022, sin excepciones) \\ 
Cobertura \emph{e-commerce} & Limitada & Óptima (API / móvil) \\ 
Riesgo de fraude & Medio-Alto (devolución) & Bajo (\textsc{SCA} + irreversibilidad) \\ 
\bottomrule
\end{tabular}



\subsection{Objetivos}
\label{subsec:Objetivos}
El \textbf{objetivo general} del Trabajo Fin de Grado es entregar un \textit{stack} de software que
simule un proveedor \emph{end-to-end} del esquema \emph{SEPA Request-to-Pay} (SRTP),
alineado con la versión 4.0 del \emph{SRTP Scheme Rulebook} \cite{epc014} y las
guías técnicas del EPC (\emph{EPC137} y \emph{EPC164}) \cite{epc137,epc164}.  
El prototipo debe ser instalable con \texttt{Docker Compose}, exponer una API
HTTP/JSON conforme a \emph{OpenAPI 3.1} y cubrir las cuatro operaciones
core \emph{(create, reject, response, cancel)} del flujo SRTP.

\noindent Para acotar y medir el trabajo se definen las siguientes \textbf{metas
específicas}:

\begin{enumerate}
  \item Implementar en \texttt{Node 20 LTS}/\texttt{Express} los \emph{endpoints}
        REST de alta disponibilidad y su contraparte \textit{callback} para
        notificaciones asíncronas (\texttt{Socket.IO}).
  \item Desarrollar un módulo de firma, sellado temporal y validación X.509
        (QSeal/QWAC) para garantizar requisitos de \emph{identificación,
        autenticación y autorización} definidos por el \emph{API Security
        Framework} \cite{epc164}.
  \item Persistir estado y auditoría en una base \texttt{SQLite} ligera
        (\texttt{SQLAlchemy}) con modelo de datos alineado a los \emph{datasets}
        DS-02, DS-07 y DS-10 del Rulebook.
  \item Entregar una colección \texttt{Postman} y un \texttt{runner} CI (GitHub Actions)
        que ejecute casos de prueba de integración, incluyendo validación de
        esquemas JSON contra \textit{schemas} oficiales del EPC.
  \item Documentar las divergencias norma $\rightarrow$ implementación y proponer
        una hoja de ruta para su homologación futura en el \emph{EDS}.
\end{enumerate}

\subsection{Fases y Métodos}
\label{subsec:FasesMetodos}
Se adopta un ciclo \textbf{ágil}, con \emph{sprints} de dos semanas y reuniones
\emph{review/retro}.  
Cada iteración termina con un incremento funcional desplegado en \texttt{Docker Hub}
y su etiqueta asociada en \texttt{Git}.

\begin{description}
  \item[Fase 1 – Análisis]%
        \begin{itemize}
        \item Lectura detallada de los documentos EPC 014, 137 y 164 y extracción
              de requisitos funcionales, de seguridad y de interoperabilidad.
        \item Modelado de actores en un diagrama de cuatro esquinas
              (Payee / Payer / PSP\textsubscript{Payee} / PSP\textsubscript{Payer}),
              identificando puntos de confianza y certificados requeridos.
        \item Priorización de \emph{user-stories} y definición de \emph{Definition of Done}.
        \end{itemize}

  \item[Fase 2 – Diseño e implementación]%
        \begin{itemize}
        \item Arquitectura \texttt{Clean Architecture} sobre \texttt{Express}:
              \texttt{routes.js}, \texttt{services.js}, \texttt{models.py}.
        \item Middleware de firma y verificación con \texttt{crypto.subtle} y
              librerías OpenSSL; generación de certificados de prueba
              \texttt{make cert}.
        \item Persistencia en \texttt{SQLite} mediante \texttt{Sequelize};
              migraciones automáticas.
        \item WebSocket \texttt{Socket.IO} encapsulado en \texttt{ext\_socketio.py}
              para notificaciones \emph{push} de estado.
        \end{itemize}

  \item[Fase 3 – Pruebas y validación]%
        \begin{itemize}
        \item Suite \texttt{Jest} + \texttt{supertest} para pruebas unitarias y de
              integración.
        \item Colección \texttt{Postman} con \emph{scripts} pre/post-request que
              firman, estampan fecha y validan contra esquemas.
        \item Inyección de fallos con \texttt{toxiproxy}: retardos, caídas de red
              y respuestas 4xx/5xx para cubrir los flujos síncrono y asíncrono.
        \item Informe de cobertura y reporte SonarQube en el pipeline CI.
        \end{itemize}
\end{description}

\subsection{Medios necesarios empleados para el desarrollo}
\label{subsec:Medios}
\begin{itemize}
  \item \textbf{Software de desarrollo:}  
        \texttt{Node.js 20 LTS}, \texttt{Express 4}, \texttt{Socket.IO 4},
        \texttt{Sequelize 6}, \texttt{Jest}, \texttt{Postman v10},
        \texttt{Docker 24}, \texttt{Docker Compose v2},
        \texttt{Git} y \texttt{GitHub Actions}.
  \item \textbf{Herramientas de apoyo:}  
        \texttt{OpenSSL 3} para gestión de certificados,
        \texttt{toxiproxy} para pruebas de resiliencia,
        \texttt{Spectral OCI} para linting de especificaciones OpenAPI.
  \item \textbf{Documentación oficial:}  
        SRTP Scheme Rulebook v4.0 \cite{epc014},  
        SRTP related API Specifications v3.1 \cite{epc137},  
        API Security Framework v2.0 \cite{epc164},  
        ISO 20022 \emph{pacs/pain/camt},  
        directivas PSD2/eIDAS.
  \item \textbf{Hardware y S.O.:}  
        Portátil x86-64, 16 GB RAM, SSD 512 GB, Ubuntu 22.04 LTS, conexión
        simétrica de 300 Mbps; virtualización \texttt{Docker Desktop}.
  \item \textbf{Repositorios y control de versiones:}  
        Organización privada en GitHub; \texttt{branch protection} y
        \texttt{semantic-versioning}.
\end{itemize}