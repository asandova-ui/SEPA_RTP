\section{Conclusiones y líneas futuras}
\label{sec:Conclusiones}

El desarrollo de este Trabajo de Fin de Grado (TFG) ha representado un esfuerzo significativo que ha culminado en la creación de un prototipo funcional de un servidor HTTP basado en el esquema Request To Pay (RTP), una iniciativa impulsada por el European Payments Council (EPC) para modernizar y agilizar los procesos de cobro en Europa. El objetivo principal del proyecto —diseñar e implementar un software que simulara las operaciones fundamentales del esquema SEPA RTP utilizando tecnologías actuales y accesibles— se ha alcanzado con éxito, lo que supone un logro tanto técnico como académico. Este trabajo no solo me ha permitido poner en práctica los conocimientos adquiridos durante mi formación en telecomunicaciones y desarrollo de software, sino también explorar un ámbito tan relevante como los sistemas de pago digitales, que desempeñan un papel crucial en la economía global de hoy.


A nivel personal, este TFG ha sido una buena oportunidad para profundizar en el ecosistema SEPA y sus diferentes esquemas de pago. En particular, he podido analizar las limitaciones del SDD, como su dependencia de plazos largos y su falta de flexibilidad para ciertos casos de uso, y contrastarlas con las ventajas que ofrece RTP, como la inmediatez en las transacciones y la mejora en la interacción entre pagadores y beneficiarios. Este aprendizaje no se ha limitado al ámbito teórico: la implementación práctica del prototipo me ha enfrentado a retos técnicos que han fortalecido mis competencias como desarrollador. Por ejemplo, integrar WebSockets para permitir una comunicación bidireccional en tiempo real entre los actores del sistema, diseñar una base de datos eficiente para almacenar y gestionar el estado de las solicitudes, y crear una interfaz de usuario sencilla pero funcional han sido tareas que me han exigido creatividad, paciencia y un enfoque metódico.


El proceso de desarrollo también ha sido una lección sobre la importancia de una metodología bien definida, lo que me permitió avanzar de manera constante, detectar errores a tiempo y ajustar los objetivos según las necesidades del proyecto. Herramientas como Postman, que utilicé para probar exhaustivamente la API del servidor, y GitHub Actions, que automatizó la integración continua, no solo facilitaron el desarrollo, sino que también me introdujeron a prácticas profesionales que son esenciales en la industria del software. Este enfoque estructurado, combinado con una documentación detallada de cada etapa, ha sido clave para mantener el control sobre el proyecto.


El sistema RTP se posiciona como un candidato ideal paraq convertirse en un estándar en la zona SEPA, especialmente a medida que más instituciones financieras y PSP adopten el esquema y lo integren en sus operaciones. En el futuro, es probable que RTP no solo facilite los pagos entre empresas y consumidores, sino que también fomente una mayor interoprabilidad y eficiencia en el mercado financiaro europeo, contribuyendo a una economía más conectada y dinámica.


\subsection{Potencial y aplicación real}
\label{subsec:Potencial}
En cuanto al prototipo desarollado, aunque satisface plenamente los objetivos establecidos para este TFG, su potencial va mucho más allá de un simple ejercicio acádemico. Durante el proceso identifiqué varias áreas de mejora que podrían transformar este simulador en una herramienta aplicable en escenarios reales. A continuación, detallo algunas de estas oportunidades de evolución:

\begin{enumerate}
    \item \textbf{Escalabilidad del sistema:} La base de datos SQLite, aunque suficiente para un prototipo, tiene limitaciones en térmninos de rendimiento y capacidad. Para soportar un mayor volumen de usuarios y transacciones, sería necesario migrar a un sistema más robusto como PostgreSQL o MySQL, que ofrecen mejor escalabilidad y soporte para entornos de producción.
    \item \textbf{Fortalecimiento de la seguridad:} El prototipo incluye medidas básicas de autenticación, pero un sistema real requeriría estándares más altos, como la encriptación de extremo a extremo, autenticación multifactor y cumplimiento con normativas como PSD2 (Payment Services Directive 2) y GDPR (General Data Protection Regulation). Estas mejoras garantizarían la protección de los datos sensibles y la confianza de los usuarios.
    \item \textbf{Interoperabilidad con sistemas bancarios:} Para que el prototipo trascienda su estado actual, sería esencial integrarlo con las APIs de bancos y PSP reales. Esto permitiría ejecutar transacciones monetarias auténticas y demostrar su utilidad en un contexto práctico, un paso crítico hacia su adopción en el mundo real.
    \item \textbf{Ampliación de funcionalidades:} El sistema podría enriquecerse con características avanzadas, como soporte para pagos recurrentes (ideal para suscripciones o facturas periódicas), compatibilidad con múltiples monedas (facilitando transacciones transfronterizas), herramientas analíticas que ofrezcan estadísticas a los usuarios, y reportes detallados sobre el historial de pagos. Estas adiciones harían que el sistema fuera más versátil y atractivo para distintos tipos de usuarios.
    \item \textbf{Mejora de la experiencia de usario:} Aunque el frontend actual es funcional, podría optimizarse con un diseño más moderno y accesible. Incorporar elementos como notificaciones push, un historial visual de transacciones, opciones de personalización y una interfaz adaptada a dispositivos móviles elevaría la usabilidad y la satisfacción del usuario.
\end{enumerate}

Estas mejoras, aunque son algo ambiciosas, son alcanzables con el tiempo y los recursos adecuados. Implementarlas no solo incrementaría la funcionalidad del prototipo, sino que también lo alinearía con las demandas de un mercado financiero en constante camnbio, donde la innovación y la adaptabilidad son esenciales.

Más allá de los aspectos técnicos, este TFG ha sido una experiencia profundamente formativa a nivel personal y profesional. Me ha enseñado a gestionar proyectos complejos, a resolver problemas de manera creativa y a valorar la importancia de la perseverancia frente a los desafíos. También me ha abierto los ojos al enorme potencial de la tecnología para transformar sectores tradicionales como el financiero, un campo que, aunque puede parecer distante de las telecomunicaciones a primera vista, está intrínsecamente conectado a través de la infraestructura digital que lo sustenta. Esta intersección entre disciplinas es, en mi opinión, uno de los aspectos más fascinantes de este proyecto y una motivación para seguir explorando esta área en el futuro.

En conclusión, este Trabajo de Fin de Grado ha sido más que un requisito académico: ha supuesto un descubrimiento, aprendizaje y crecimiento. El esquema Request To Pay, con su enfoque innovador y su capacidad para simplificar los procesos de cobro, tiene el potencial de redefinir cómo interactuamos con los servicios financieros en Europa y más allá. El prototipo que he desarrollado, aunque es limitado en su alcance actual, es una prueba tangible de ese potencial. Con las mejoras adecuadas, podría evolucionar para convertirse en una solución práctica y relevante, contribuyendo a un ecosistema de pagos más eficiente, seguro y accesible. Estoy contento con el trabajo realizado y estoy seguro de que este sistema marcará el futuro en los pagos digitales.

