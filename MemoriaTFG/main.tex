\documentclass{article}
\usepackage{graphicx} % Required for inserting images
\usepackage[spanish, english]{babel}
\usepackage[hidelinks]{hyperref}
\usepackage{makecell}
\usepackage[numbered,framed]{listings}
\usepackage{color} % red, green, blue, yellow, cyan, magenta, black, white
\usepackage{enumitem}
\usepackage{lmodern}
\usepackage{array}
\usepackage{float}
\usepackage{tabularx}
\usepackage{siunitx}
\usepackage{booktabs}
\usepackage{footnote}          % <— nuevo
\usepackage{siunitx}
\usepackage{threeparttable}
\usepackage{tikz}       % Diagrama en capas (cubo)
\usepackage{footmisc}   % Mejor gestión de notas a pie de página (opcional)
\definecolor{mylilas}{RGB}{170,55,241}
\definecolor{backcolour}{rgb}{0.92,0.95,0.95}

\usepackage{longtable}
\usepackage{geometry}
\usepackage{lipsum}

\usepackage{listings}
\usepackage{xcolor}
\lstdefinestyle{custom}{
    backgroundcolor=\color{black!5},           % Fondo más sutil (gris muy claro)
    basicstyle=\ttfamily\footnotesize,         % Fuente monoespaciada más pequeña y profesional
    keywordstyle=\color{blue!70!black},        % Palabras clave en azul oscuro
    stringstyle=\color{orange!80!black},       % Cadenas en naranja suave
    commentstyle=\color{gray!50},              % Comentarios en gris claro
    numberstyle=\tiny\color{gray!40},          % Números de línea discretos
    numbers=left,                              % Números a la izquierda
    numbersep=10pt,                            % Espaciado entre números y código
    breaklines=true,                           % Romper líneas largas
    breakatwhitespace=true,                    % Romper solo en espacios
    frame=tb,                                  % Marco solo arriba y abajo (más elegante)
    framesep=5pt,                              % Espaciado interno del marco
    captionpos=b,                              % Título debajo
    showstringspaces=false,                    % No mostrar espacios en cadenas
    tabsize=4,                                 % Tamaño de tabulación
    escapeinside={(*@}{@*)},                   % Permitir escapar a LaTeX dentro del código
    morekeywords={app, def, return, send, register_blueprint, SocketIO} % Palabras clave adicionales
}




\begin{document}



% Configurando la numeración arábiga desde el inicio
\pagenumbering{arabic}

% Creando la portada
\begin{titlepage}
    \centering
    {\includegraphics[width=0.9\textwidth]{Imagenes/Universidad-de-Valladolid.png}}\par
    {\bfseries\Large Escuela Técnica Superior de Ingenieros de Telecomunicación\par}
    \vspace{0.5cm}
    {\bfseries\itshape\Large Trabajo de Fin de Grado \par}
    \vspace{0.5cm}
    {\scshape Grado en Ingeniería de Tecnologías de Telecomunicación \par}
    \vspace{0.5cm}
    {\bfseries\scshape\Large Implementación de un Servidor HTTP para Emular un Prototipo del Sistema Request To Pay del EPC \par}
    \vspace{1.5cm}
    { Autor: Alonso\\}
    { Sandoval Martínez \par}
    { Tutores:\\}
    { D. Tutor1 \\ D. Tutor2 \par}
    \vspace{0.5cm} {Valladolid, MES 202X \par}
\end{titlepage}

% Configurando el idioma
\selectlanguage{spanish}  

% Añadiendo el índice justo después de la portada
\newpage
\tableofcontents

% Añadiendo los agradecimientos desde un archivo externo
\newpage
\phantomsection
\addcontentsline{toc}{section}{Agradecimientos}
\section*{Agradecimientos}
Agradezco a mis tutores, familia y amigos por su apoyo durante la realización de este trabajo. Su orientación y motivación han sido fundamentales para completar este proyecto.

% Añadiendo el resumen en español
\newpage
\phantomsection
\addcontentsline{toc}{section}{Resumen}
\begin{abstract}
    Este trabajo presenta la implementación de Como se muestra en \cite{smith2020iapago}, el protocolo SEPA permite...
    un servidor HTTP para emular un prototipo del sistema Request To Pay desarrollado por el European Payments Council (EPC), abordando...
\end{abstract}
\noindent {\small \textbf{\textit{Palabras clave---}} Servidor HTTP, Request To Pay, EPC, Telecomunicaciones, Pagos Electrónicos}

% Volviendo al español para el resto del documento
\selectlanguage{spanish} 


% Comenzando el cuerpo principal (numeración ya es arábiga)
\newpage

\section{Introducción}
\label{sec:Introduccion}

Para comprender el entorno actual de los pagos en Europa, conviene arrancar por la \emph{Single Euro Payments Area} (SEPA): un espacio comunitario en el que todos los pagos en euros se rigen por los mismos estándares técnicos y normas operativas, de modo que enviar dinero de un país a otro resulta tan ágil y claro como una transferencia nacional. SEPA estableció protocolos de mensajería comunes, armonizó los plazos de liquidación y fijó reglas uniformes de protección al usuario, creando la base sobre la que se despliegan hoy los servicios de pago más innovadores.

En los últimos diez años, la digitalización de los servicios financieros ha cambiado por completo cómo particulares y empresas gestionan sus transacciones dentro de ese marco SEPA. Las transferencias instantáneas, las API abiertas de los bancos y el auge del comercio electrónico han disparado la demanda de procesos de cobro que sean sencillos, transparentes y en tiempo real. No obstante, los instrumentos de pago tradicionales —tarjetas, transferencias convencionales o domiciliaciones— nacieron en un contexto muy distinto y todavía arrastran limitaciones que penalizan tanto la experiencia de usuario como la eficiencia operativa.

Aquí es donde entra en juego \emph{Request-to-Pay} (RTP). Este servicio de mensajería permite al beneficiario enviar al pagador una solicitud de pago digital estructurada, con todos los detalles (importe, concepto, vencimiento), y recibir en segundos una respuesta —aceptación, rechazo o aplazamiento— antes de iniciar el movimiento de fondos. RTP no sustituye los métodos de pago existentes, sino que actúa como una capa de orquestación sobre la infraestructura SEPA (y, en especial, los pagos inmediatos) y los canales de banca online, facilitando la conciliación, reduciendo la fricción en el cobro y modernizando la experiencia tanto para empresas como para consumidores.

\subsection{Motivación}
\label{subsec:Motivacion}

La domiciliación bancaria regulada por el esquema \textit{SEPA Direct Debit} (\textbf{SDD})\footnote{\textit{SEPA Direct Debit} es el instrumento paneuropeo de cargo en cuenta regulado por el \emph{European Payments Council}.} desde 2014, sigue siendo el método principal para cobros recurrentes en España. No obstante, su estructura, pensada para un entorno de procesos \emph{offline}—genera hoy inconvenientes que chocan con las demandas de inmediatez, seguridad y experiencia de usuario fluida que caracterizan la economía digital actual.

\subsubsection{Ineficiencias operativas detectadas}

Tras analizar la operativa SDD nacional se han detectado una serie de ineficiencias resumidas en los siguientes \textbf{5} puntos:
\begin{enumerate}[label=\textbf{\arabic*.}, leftmargin=0.75cm]
  \item \textbf{Modelo off-line y necesidad de mandato físico}\\
        \begin{itemize}[leftmargin=0.45cm]
            \item El pago se inicia sin interacción \emph{online}; El deudor debe firmar y remitir \textbf{mandato SEPA}\footnote{Autorización que un deudor otroga a un acreedor para permitir el cobro automático de pagos mediante SDD.} y el acreedor debe custodiarlo durante el periodo de vida del contrato y hasta 14 meses tras la última transacción.
            \item No existe un estándar eMandate\footnote{Un eMandate es la versión digital de un mandato.} interoperable, cada banco utiliza su propio estándar.
        \end{itemize}
        \vspace{-0.1cm}
        \textit{Consecuencias}: Fricción en la venta digital, costes de back-office (archivado, auditorías y gestión) y riesgo legal si el mandato no aparece ante una devolución.

  \item \textbf{Derecho a devolución prolongado}\\
        \begin{itemize}[leftmargin=0.45cm]
            \item \emph{Cobro autorizado}: el deudor dispone de \textbf{8~semanas} para devolver sin causa (\emph{“no-questions-asked”}).%
            \item \emph{Cobro no autorizado}: hasta \textbf{13~meses} para reclamar si el banco emisor no puede demostrar mandato válido.%
        \end{itemize}
        \vspace{-0.1cm}
        \textit{Consecuencias}: gran incertidumbre sobre la firmeza del ingreso, reservas de liquidez, provisiones contables y alto nivel de \emph{friendly fraud}\footnote{servicio consumido y posterior devolución.}.

  \item \textbf{Ciclos de cobro lentos}\\
        \begin{itemize}[leftmargin=0.45cm]
            \item En esquema \textsc{Core}\footnote{CORE es el esquema estándar de domiciliación bancaria SEPA para pagos entre empresas y consumidores.}: envío al banco \textbf{D-5} para primera domiciliación y \textbf{D-2} para recurrencias; liquidación interbancaria \textbf{+2~días}.
            \item 6–8~días naturales entre petición y abono firme, incompatibles con venta inmediata o entrega digital.
        \end{itemize}
        \vspace{-0.1cm}
        \textit{Consecuencias}: Tesorería imprevisible (cash-flow) y riesgo de prestar servicio sin cobro confirmado.

  \item \textbf{Costes y complejidad de las R-transactions\footnote{ transacciones de rechazo, devolución o reembolso asociadas a pagos fallidos o no autorizados.}}\\
        \begin{itemize}[leftmargin=0.45cm]
            \item Existen distintos códigos de R-transactions y cada uno de ellos implica un flujo distinto, lo que dificulta la gestión.
        \end{itemize}
        \vspace{-0.1cm}
        \textit{Consecuencias}: Necesidad de equipos específicos de conciliación y recobro, lo que conlleva un coste directo y pérdida de productividad.

  \item \textbf{Ausencia de autorización fuerte (\textsc{SCA})\footnote{La SCA (Strong Customer Authentication) es un mecanismo de seguridad que exige verificar al usuario con al menos dos factores.}}\\
        \begin{itemize}[leftmargin=0.45cm]
            \item El SDD se basa en consentimiento previo, no aplica SCA al momento del cargo.
        \end{itemize}
        \vspace{-0.1cm}
        \textit{Consecuencias}: Mayor riesgo de cargos disputados y se desaprovechan métodos de identidad digital ya existentes.
\end{enumerate}

\paragraph{En conclusión.} Estas ineficiencias se traducen en (i) estructura de costes elevada por devoluciones y personal especializado; (ii) liquidez incierta—los ingresos se confirman con días de retraso y pueden desaparecer meses después—y (iii) freno a la economía digital online, incapaz de ofrecer experiencias de pago instantáneas comparables a tarjeta, monederos o Bizum.

\subsubsection{Oportunidad de un esquema Request-to-Pay}
El estándar \textit{SEPA Request-to-Pay} (\textbf{SRTP})\footnote{Iniciativa del European Payments Council que define un flujo de solicitud (\textit{request}) y aceptación de pagos en tiempo real, apoyado en mensajería \textbf{ISO 20022} (p.\,ej.\, \texttt{pain.013}/\texttt{pain.014}) y agnóstico respecto al instrumento de liquidación posterior.} se perfila como la evolución natural de la domiciliación SEPA. Sus ventajas técnicas frente al SDD son:

\begin{enumerate}[label=\alph*)]
  \item \textbf{Autenticación reforzada y consentimiento digital inmediato}\\
        El acreedor emite un request que el deudor aprueba en su banca o \emph{wallet} mediante \textsc{SCA}. Este gesto sustituye al mandato físico y genera una prueba electrónica de consentimiento, firmada y almacenada dentro del PSP del pagador.

  \item \textbf{Irrevocabilidad y mitigación de fraude \emph{post-servicio}}\\
        Tras la aceptación, el pago se realiza mediante SCT Inst\footnote{\textsc{SCT Inst}: transferencia inmediata SEPA con liquidación ≤10 s.}. Al no existir derecho de devolución automática, se elimina el \emph{friendly fraud} asociado a la devolución de recibos y se reducen provisiones por impago.

  \item \textbf{Liquidez \emph{real-time} y conciliación automática}\\
        La disponibilidad de fondos en \(\leq10\,\text{s}\) permite planificar tesorería al instante. Los identificadores de extremo a extremo y las referencias estructuradas ISO 20022 se transmiten sin perderse entre sistemas, de modo que la conciliación queda totalmente automatizada.

  \item \textbf{Simplificación operativa}\\
        Desaparecen las \textsc{R-transactions}, la custodia de mandatos y las tareas de back-office. El flujo se limita a dos mensajes (request y aceptación) y, opcionalmente, una transferencia instantánea, con clara trazabilidad extremo a extremo.

  \item \textbf{Flexibilidad comercial y costes reducidos}\\
        SRTP admite cobros únicos, recurrentes y fraccionados desde web o app vía enlaces profundos, QR o API, y al ser un pago mediante SCT Inst, las comisiones bancarias son muy inferiores a las de tarjeta o a las de gestión de devoluciones SDD.
\end{enumerate}

En síntesis, SRTP traslada las ventajas históricas de la domiciliación—bajo coste y cobertura paneuropea—al entorno digital e inmediato, resolviendo los puntos técnicos que hoy limitan la competitividad del SDD en España y en la zona SEPA.
\vspace{0.5cm}

\renewcommand{\arraystretch}{1.3}
\begin{tabular}{@{}p{4.5cm}p{4.1cm}p{4.1cm}@{}}
\toprule
\textbf{Aspecto} & \textbf{SDD} & \textbf{SRTP (+ SCT Inst)} \\ \midrule
Autorización & Mandato off-line & Consentimiento digital (\textsc{SCA}) en tiempo real \\ 
Plazo de devolución & 8\,semanas / 13\,meses & No aplica (operación irrevocable) \\ 
Disponibilidad de fondos & 5–8 días & Segundos (\textless10\,s) \\ 
Coste operativo & Alto (mandatos, \textsc{R-codes}) & Bajo (mensajería ISO 20022, sin excepciones) \\ 
Cobertura \emph{e-commerce} & Limitada & Óptima (API / móvil) \\ 
Riesgo de fraude & Medio-Alto (devolución) & Bajo (\textsc{SCA} + irreversibilidad) \\ 
\bottomrule
\end{tabular}



\subsection{Objetivos}
\label{subsec:Objetivos}
El \textbf{objetivo general} del Trabajo Fin de Grado es entregar un \textit{stack} de software que
simule un proveedor \emph{end-to-end} del esquema \emph{SEPA Request-to-Pay} (SRTP),
alineado con la versión 4.0 del \emph{SRTP Scheme Rulebook} \cite{epc014} y las
guías técnicas del EPC (\emph{EPC137} y \emph{EPC164}) \cite{epc137,epc164}.  
El prototipo debe ser instalable con \texttt{Docker Compose}, exponer una API
HTTP/JSON conforme a \emph{OpenAPI 3.1} y cubrir las cuatro operaciones
core \emph{(create, reject, response, cancel)} del flujo SRTP.

\noindent Para acotar y medir el trabajo se definen las siguientes \textbf{metas
específicas}:

\begin{enumerate}
  \item Implementar en \texttt{Node 20 LTS}/\texttt{Express} los \emph{endpoints}
        REST de alta disponibilidad y su contraparte \textit{callback} para
        notificaciones asíncronas (\texttt{Socket.IO}).
  \item Desarrollar un módulo de firma, sellado temporal y validación X.509
        (QSeal/QWAC) para garantizar requisitos de \emph{identificación,
        autenticación y autorización} definidos por el \emph{API Security
        Framework} \cite{epc164}.
  \item Persistir estado y auditoría en una base \texttt{SQLite} ligera
        (\texttt{SQLAlchemy}) con modelo de datos alineado a los \emph{datasets}
        DS-02, DS-07 y DS-10 del Rulebook.
  \item Entregar una colección \texttt{Postman} y un \texttt{runner} CI (GitHub Actions)
        que ejecute casos de prueba de integración, incluyendo validación de
        esquemas JSON contra \textit{schemas} oficiales del EPC.
  \item Documentar las divergencias norma $\rightarrow$ implementación y proponer
        una hoja de ruta para su homologación futura en el \emph{EDS}.
\end{enumerate}

\subsection{Fases y Métodos}
\label{subsec:FasesMetodos}
Se adopta un ciclo \textbf{ágil}, con \emph{sprints} de dos semanas y reuniones
\emph{review/retro}.  
Cada iteración termina con un incremento funcional desplegado en \texttt{Docker Hub}
y su etiqueta asociada en \texttt{Git}.

\begin{description}
  \item[Fase 1 – Análisis]%
        \begin{itemize}
        \item Lectura detallada de los documentos EPC 014, 137 y 164 y extracción
              de requisitos funcionales, de seguridad y de interoperabilidad.
        \item Modelado de actores en un diagrama de cuatro esquinas
              (Payee / Payer / PSP\textsubscript{Payee} / PSP\textsubscript{Payer}),
              identificando puntos de confianza y certificados requeridos.
        \item Priorización de \emph{user-stories} y definición de \emph{Definition of Done}.
        \end{itemize}

  \item[Fase 2 – Diseño e implementación]%
        \begin{itemize}
        \item Arquitectura \texttt{Clean Architecture} sobre \texttt{Express}:
              \texttt{routes.js}, \texttt{services.js}, \texttt{models.py}.
        \item Middleware de firma y verificación con \texttt{crypto.subtle} y
              librerías OpenSSL; generación de certificados de prueba
              \texttt{make cert}.
        \item Persistencia en \texttt{SQLite} mediante \texttt{Sequelize};
              migraciones automáticas.
        \item WebSocket \texttt{Socket.IO} encapsulado en \texttt{ext\_socketio.py}
              para notificaciones \emph{push} de estado.
        \end{itemize}

  \item[Fase 3 – Pruebas y validación]%
        \begin{itemize}
        \item Suite \texttt{Jest} + \texttt{supertest} para pruebas unitarias y de
              integración.
        \item Colección \texttt{Postman} con \emph{scripts} pre/post-request que
              firman, estampan fecha y validan contra esquemas.
        \item Inyección de fallos con \texttt{toxiproxy}: retardos, caídas de red
              y respuestas 4xx/5xx para cubrir los flujos síncrono y asíncrono.
        \item Informe de cobertura y reporte SonarQube en el pipeline CI.
        \end{itemize}
\end{description}

\subsection{Medios necesarios empleados para el desarrollo}
\label{subsec:Medios}
\begin{itemize}
  \item \textbf{Software de desarrollo:}  
        \texttt{Node.js 20 LTS}, \texttt{Express 4}, \texttt{Socket.IO 4},
        \texttt{Sequelize 6}, \texttt{Jest}, \texttt{Postman v10},
        \texttt{Docker 24}, \texttt{Docker Compose v2},
        \texttt{Git} y \texttt{GitHub Actions}.
  \item \textbf{Herramientas de apoyo:}  
        \texttt{OpenSSL 3} para gestión de certificados,
        \texttt{toxiproxy} para pruebas de resiliencia,
        \texttt{Spectral OCI} para linting de especificaciones OpenAPI.
  \item \textbf{Documentación oficial:}  
        SRTP Scheme Rulebook v4.0 \cite{epc014},  
        SRTP related API Specifications v3.1 \cite{epc137},  
        API Security Framework v2.0 \cite{epc164},  
        ISO 20022 \emph{pacs/pain/camt},  
        directivas PSD2/eIDAS.
  \item \textbf{Hardware y S.O.:}  
        Portátil x86-64, 16 GB RAM, SSD 512 GB, Ubuntu 22.04 LTS, conexión
        simétrica de 300 Mbps; virtualización \texttt{Docker Desktop}.
  \item \textbf{Repositorios y control de versiones:}  
        Organización privada en GitHub; \texttt{branch protection} y
        \texttt{semantic-versioning}.
\end{itemize}
\newpage

\section{Antecedentes y estado del arte}
\label{sec:Antecedentes}
Esta sección proporciona una visión general del contexto técnico y normativo del proyecto...

\subsection{Visión general del sistema \textit{Request To Pay}}
\label{subsec:VisionRTP}
El sistema \textit{Request To Pay} es una iniciativa del EPC para facilitar los pagos electrónicos...

\subsection{Tecnologías de servidores HTTP}
\label{subsec:TecnologiasHTTP}
Se revisan las principales tecnologías y frameworks utilizados para desarrollar servidores HTTP...

\subsection{Prototipos y estándares del EPC}
\label{subsec:PrototiposEPC}
El EPC ha establecido una serie de estándares que se tomaron como referencia para este proyecto...

\newpage

% 3. Diseño e Implementación
\section{Diseño e Implementación}
\label{sec:DisenoImplementacion}
Para implementar un sistema completo de \textit{Request To Pay}, se ha desarrollado un servidor HTTP que expone una API RESTful y un cliente web en tiempo real. El objetivo de este apartado es detallar cómo se ha llevado a cabo esta implementación, así como las fases de desarrollo involucradas.

\subsection{Fundamentos Teóricos}
Antes de profundizar en la implementación, es fundamental aclarar los dos pilares teóricos en los que se basa el sistema:

\begin{itemize}
    \item \textbf{Servidor HTTP}: Una aplicación que permanece a la escucha en un puerto de red, esperando conexiones TCP de clientes capaces de comunicarse mediante el protocolo HTTP. Una vez establecida la conexión, el servidor recibe un mensaje estructurado con los siguientes componentes:
    \begin{itemize}
        \item \textit{Línea de petición}: Indica el método (GET, POST, etc.), la ruta del recurso solicitado y la versión del protocolo.
        \item \textit{Cabeceras}: Aportan metadatos, como el tipo de contenido, credenciales o longitud del mensaje.
        \item \textit{Cuerpo} (opcional): Contiene datos adicionales, si el método lo requiere.
    \end{itemize}
    El servidor interpreta la ruta, determina qué componente interno debe procesarla, ejecuta la lógica correspondiente y genera una respuesta formada por:
    \begin{itemize}
        \item \textit{Línea de estado}: Incluye un código de resultado (por ejemplo, 200 OK, 404 Not Found, 500 Internal Server Error).
        \item \textit{Cabeceras}: Describen la respuesta.
        \item \textit{Cuerpo} (opcional): Contiene datos, como HTML o JSON.
    \end{itemize}
    Tras enviar la respuesta, la conexión puede cerrarse o mantenerse activa para futuras peticiones, dependiendo de la versión del protocolo y las cabeceras de control. En esencia, el servidor HTTP funciona como el centro de operaciones que recibe todas las solicitudes y coordina el acceso a la lógica y los datos de la aplicación.

    \item \textbf{API RESTful}: Se basa en los principios de la arquitectura REST (\textit{Representational State Transfer}\footnote{Transferencia de Estado Representacional}) aplicados al protocolo HTTP para exponer recursos de forma uniforme y predecible. Sus características principales son:
    \begin{itemize}
        \item Cada entidad del dominio (por ejemplo, un usuario o una petición RTP) se representa mediante una URL estable.
        \item Los verbos HTTP (POST, GET, PUT, DELETE) describen operaciones como creación, consulta, modificación o eliminación de recursos.
        \item El servidor es \textit{sin estado}, por lo que cada solicitud contiene toda la información necesaria, facilitando la escalabilidad horizontal.\footnote{Es decir, permite añadir o retirar servidores sin necesidad de compartir sesiones en memoria.}
        \item La uniformidad de los códigos de estado y los formatos de representación asegura que clientes heterogéneos consuman la API de manera predecible.
        \item Herramientas como cachés, control de versiones en URLs o cabeceras, y negociación de contenido permiten evolucionar la interfaz sin afectar a los consumidores existentes.
    \end{itemize}
\end{itemize}

En conjunto, el servidor HTTP actúa como el camino por donde viajan las solicitudes, mientras que la API RESTful establece las reglas claras y fáciles de mantener para que ese camino conecte eficientemente la lógica del servidor con los diversos clientes que dependen de ella.
%%%%%%%%%%%%%%%%%%%%%%%%%%%%%%%%%%%%%%%%%%%%%%%%%%%%%%%%%%%%%%%%%%%%%%%%%%%%%%%%%%%%%%%%%%%%%%%%%%%%%%%%%%%%%%%%%%
\subsection{Tecnologías utilizadas}
\label{subsec:Herramientas de desarrollo}
El desarrollo del proyecto se ha llevado a cabo en un entorno virtual Python que aísla las dependencias y permite reproducir la instalación mediante el comando \texttt{pip install -r requirements.txt}. Este enfoque asegura consistencia y portabilidad, facilitando tanto la colaboración entre desarrolladores como el despliegue en entornos diversos.

\textbf{Backend.} El núcleo del backend se fundamenta en Python 3 y Flask, un micro-framework que proporciona un servidor WSGI, un sistema de enrutado eficiente y una integración fluida con el estándar HTTP. Sobre esta base, se incorpora Flask-socketIO, un middleware que habilita la negociación de WebSockets, permitiendo servir tráfico HTTP y comunicación bidireccional en el mismo puerto de manera transparente. Para la gestión de datos, se emplea SQLAlchemy a través de Flask-SQLAlchemy, lo que facilita la representación de modelos de dominio como objetos Python, mientras que SQLite actúa como una base de datos ligera y autónoma durante el desarrollo, eliminando la necesidad de un servidor externo para transacciones básicas. Este núcleo se complementa con utilidades de la librería estándar de Python, incluyendo:
\begin{itemize}
    \item \texttt{hashlib}, para firmar transacciones de estado mediante el algoritmo SHA-256;
    \item \texttt{datetime}, para registrar marcas temporales en los logs.
\end{itemize}

\textbf{Frontend.} En el lado del cliente, se ha implementado un frontend estático basado en HTML5, CSS3 y JavaScript ES6, priorizando la ligereza al evitar frameworks complejos. La maquetación adaptativa se logra mediante Bootstrap 5, cargado vía CDN, y la iconografía se enriquece con Font Awesome, también distribuido por CDN. La comunicación en tiempo real se establece con el cliente Socket.IO 4.x, que conecta vía WebSocket al mismo host y puerto que Flask, mientras que las peticiones REST se realizan de forma nativa con la Fetch API, sin depender de librerías adicionales.

\textbf{Herramientas de desarrollo.} El entorno de trabajo se ha centrado en Visual Studio Code como editor principal, aprovechando sus extensiones para optimizar la gestión del proyecto y el desarrollo del código. El control de versiones se ha gestionado con Git, utilizando ramas específicas para cada funcionalidad nueva, lo que asegura un desarrollo ordenado y trazable. Para las pruebas manuales de la API, se ha empleado Postman, donde se diseñó una colección de peticiones parametrizadas que serán detalladas en secciones posteriores del documento.

El uso de un entorno virtual junto con dependencias consolidadas establece una base robusta para el despliegue de la aplicación en entornos más exigentes, como contenedores, nubes públicas o servidores locales, preservando la integridad de su arquitectura fundamental.
%%%%%%%%%%%%%%%%%%%%%%%%%%%%%%%%%%%%%%%%%%%%%%%%%%%%%%%%%%%%%%%%%%%%%%%%%%%%%%%%%%%%%%%%%%%%%%%%%%%%%%%%%%%%%%%%%%%%%%%%%%%%%%%
\subsubsection{Estructura y funcionamiento}
\label{subsubsec:EstructuraFuncionamiento}
El código del proyecto se estructura en dos componentes principales, \textbf{backend} y \textbf{frontend}, interconectados mediante los protocolos HTTP y WebSockets. Esta división, diseñada de manera intencionada, responde a la necesidad de lograr un desarrollo ordenado, eficiente y preparado para futuros crecimientos, considerando que trabajé en el proyecto de forma individual. Al separar el \textit{backend}, encargado de la lógica de negocio, el manejo de datos y la comunicación con el cliente, del \textit{frontend}, centrado en la interfaz de usuario y la experiencia interactiva, se obtiene una arquitectura clara y modular. Esta organización aporta múltiples ventajas: mejora la mantenibilidad al permitir identificar y corregir errores de manera localizada, simplifica la incorporación de nuevas funcionalidades sin alterar otras partes del sistema y refleja fielmente la arquitectura empleada en entornos de producción reales, lo que facilita una transición fluida hacia despliegues en contenedores, nubes públicas o servidores locales. Además, esta separación promueve la reutilización de código, ya que el \textit{backend} puede servir a múltiples clientes (como aplicaciones móviles o de escritorio) y el \textit{frontend} puede adaptarse a diferentes dispositivos sin modificar la lógica subyacente. A continuación, se describen los archivos que componen cada una de estas partes y su rol específico en el proyecto.










\subsubsection{Protocolos y estándares implementados}
\label{subsubsec:Protocolos}
Se implementaron protocolos como HTTP/1.1 y se consideraron estándares de seguridad...

\subsection{Emulación del prototipo \textit{Request To Pay}}
\label{subsec:EmulacionRTP}
\subsubsection{Diseño de las funcionalidades}
\label{subsubsec:DisenoFuncionalidades}
Se replicaron las funcionalidades clave del sistema \textit{Request To Pay}...

\subsubsection{Modelos de procesamiento de solicitudes}
\label{subsubsec:ProcesamientoSolicitudes}
La lógica de negocio se diseñó para procesar solicitudes de pago...

\subsection{Herramientas de desarrollo}
\label{subsec:HerramientasDesarrollo}
\subsubsection{Lenguajes y frameworks}
\label{subsubsec:LenguajesFrameworks}
Se utilizó [especificar lenguaje/framework, e.g., Node.js] para el desarrollo...

\subsubsection{Pruebas y validación}
\label{subsubsec:PruebasValidacion}
Se emplearon herramientas como Postman para validar la funcionalidad del servidor...

\newpage

% 4. Métodos
%\section{Métodos}
\label{sec:Metodos}
Esta sección describe los métodos empleados para desarrollar y evaluar el sistema...

\subsection{Preparación de los datos de entrada}
\label{subsec:PreparacionDatos}
\subsubsection{Simulación de solicitudes HTTP}
\label{subsubsec:SimulacionSolicitudes}
Se simularon solicitudes HTTP utilizando herramientas como curl...

\subsubsection{Configuración de escenarios de prueba}
\label{subsubsec:EscenariosPrueba}
Se definieron escenarios de prueba que cubren casos de uso típicos...

\subsection{Evaluación del sistema}
\label{subsec:EvaluacionSistema}
\subsubsection{Métricas de rendimiento}
\label{subsubsec:MetricasRendimiento}
Se evaluaron métricas como latencia y \textit{throughput}...

\subsubsection{Diseño de pruebas}
\label{subsubsec:DisenoPruebas}
Se diseñaron pruebas unitarias, de integración y de carga...

\subsubsection{Análisis de resultados}
\label{subsubsec:AnalisisResultados}
Los resultados obtenidos se analizaron para identificar cuellos de botella...

\subsection{Optimización del servidor}
\label{subsec:Optimizacion}
Se implementaron mejoras para optimizar el rendimiento del servidor...

\subsection{Validación cruzada de la implementación}
\label{subsec:ValidacionCruzada}
La implementación se comparó con los estándares del EPC...

\newpage

% 5. Resultados y discusión
%\section{Resultados y discusión}
\label{sec:ResultadosDiscusion}
Esta sección presenta los resultados obtenidos y su análisis...

\subsection{Presentación de los resultados}
\label{subsec:PresentacionResultados}
\subsubsection{Comparativa con los estándares del EPC}
\label{subsubsec:ComparativaEPC}
La implementación cumple con los requisitos establecidos por el EPC...

\subsubsection{Análisis de casos de uso}
\label{subsubsec:AnalisisCasosUso}
Se analizaron casos de uso reales para evaluar el comportamiento del sistema...

\subsection{Discusión de los resultados}
\label{subsec:DiscusionResultados}
Se discuten las implicaciones de los resultados y las limitaciones encontradas...

\newpage
% 6. Conclusiones y líneas futuras
%\section{Conclusiones y líneas futuras}
\label{sec:Conclusiones}
En esta sección se resumen los logros y se proponen mejoras futuras...

\subsection{Conclusiones}
\label{subsec:ConclusionesSub}
El proyecto logró implementar con éxito un servidor HTTP para \textit{Request To Pay}...

\subsection{Limitaciones y Líneas Futuras}
\label{subsec:LimitacionesFuturas}
Entre las limitaciones se encuentra la escalabilidad del sistema, que podría mejorarse...

\newpage
% Añadiendo los apéndices
\clearpage
\appendix
\section{Códigos}
\label{sec:apendice}
    \lstinputlisting[caption={Código de ejemplo.}, captionpos=b]{Codes/Example.m}

% Añadiendo las referencias
\clearpage
\renewcommand{\bibname}{Referencias}
\bibliographystyle{castellano2}
\bibliography{bibliografia}

% Añadiendo el índice de figuras
\newpage
\renewcommand{\listfigurename}{Índice de Figuras}
\listoffigures
\addcontentsline{toc}{section}{Índice de Figuras}

% Añadiendo el índice de tablas
\newpage
\renewcommand{\listtablename}{Índice de Tablas}
\listoftables
\addcontentsline{toc}{section}{Índice de Tablas}

% Añadiendo el índice de códigos
\newpage
\renewcommand*{\lstlistlistingname}{Índice de Códigos}
\renewcommand*{\lstlistingname}{Código}
\lstlistoflistings
\addcontentsline{toc}{section}{Índice de Códigos}

\end{document}